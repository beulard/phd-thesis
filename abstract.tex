\chapter*{Abstract}


COMET is a future high-precision experiment searching for charged lepton flavour
violation through the muon-to-electron conversion process. It aims to push the
intensity frontier of particle physics by coupling an intense muon beam with
cutting-edge detector technology. The first stage of the experiment, COMET
Phase\nobreakdash-I, is currently being assembled and will soon enter its data acquisition
period. It plans to achieve a single event sensitivity to $\mu$--$e$ conversion
in aluminium of $3.1 \times 10^{-15}$.

This thesis presents a study of the sensitivity and backgrounds of \mbox{COMET
Phase\nobreakdash-I} using the latest Monte Carlo simulation data produced. The background
contribution from cosmic ray-induced atmospheric muons is estimated using a
backward Monte Carlo approach, which allows computational resources to be
focused on the most critical signal-mimicking events.

Analysis of a $\mu$--$e$ conversion simulation sample suggests that COMET
Phase\nobreakdash-I will reach a single event sensitivity of $3.6 \times 10^{-15}$ within 146
days of data acquisition. 
% TODO replace with total selected count, recommended CRV+PID effs.
In that period, the background contribution from atmospheric muons is estimated
to be 0.08 events under optimistic conditions, namely a
\SI{99.99}{\percent}-efficient Cosmic Ray Veto system and \SI{99}{\percent} muon
identification rate by the main detector.