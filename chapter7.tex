\chapter{Discussion}

This thesis presents two main original contributions: we designed and built a
fake MC data generator using machine learning, and we estimated the rate of
atmospheric muon backgrounds in the COMET Phase-I $\mu$--$e$ conversion search
using a backward MC simulation technique.

\section{Data augmentation with machine learning}
% Summary
\subsection{Summary}
The CDC GAN, discussed in Chapter~\ref{ch:gan}, provides a way to produce
synthetic Monte Carlo data in the Cylindrical Drift Chamber. The design of this
pair of neural networks is based on Generative Adversarial Networks, a technique
widely used in HEP and in other domains to produce fake samples by learning from
a dataset of real samples. The CDC GAN is trained on a dataset of MC-simulated
hits, using the WGAN-GP algorithm. Its discriminator and generator are
convolutional neural networks that can process ordered sequences of hits, which
allows the GAN to learn both from per-hit features and from the relations
between multiple consecutive hits. 
Once trained, the GAN is evaluated in multiple ways. First, the distributions of
the synthetic data are compared to those of the training dataset. Then, quality
metrics are investigated, which typically involve an external performance criterion
implemented by a third neural network trained independently.

% Interpretation
%\subsection{Interpretation}
After training, the CDC GAN provides a way to produce original hit data $10^7$
times more quickly than by means of traditional MC simulation. We can therefore
generate large samples of background events very efficiently. This can be
useful, for instance, in studies which require a pure signal sample to be
overlaid onto an ambient background. Alternatively, GAN-generated data can serve
as a part of a mock dataset, which combines all sources of backgrounds into one
large-scale sample, of a size comparable to the amount of data collected during
the experiment.


% Implications (relevance vs literature)
The CDC GAN % is specific to the CDC, but the methods used can be
% generalised to other detector systems in COMET as well as other experiments.

% Maybe talk about:
One of the difficulties in designing the GAN was in determining the best way
to handle the discrete feature, wire index. For the CDC GAN, we chose to
represent the wire index as a one-hot encoded vector. However, other methods
were investigated, such as using the actual or transformed wire positions, or
using a trainable embedding matrix to make the GAN learn its preferred
representation of the geometry.



% Limitations
\subsection{Limitations}
% + CDC GAN can only produce a part of the whole: noise-like hit data only.
The CDC GAN is only trained on a selected part of the CDC hit data.
Reconstructible tracks, defined by their momentum $p>\SI{50}{\MeV/\clight}$, are
removed from the training dataset of the GAN in order to prevent it from
generating signal-like events. This is done because the GAN generates unlabelled
hits, and does not provide truth information such as parent particle and
momentum, which is required to perform, for instance, background contamination
studies.
Only hits from particles that have a momentum lower than \SI{50}{\MeV/\clight}
are learned from. Although this is preferable to avoid biasing samples, it
prevents us from using the GAN to generate an entire mock dataset: the GAN must
be complemented by real MC hits of reconstructible tracks.
In addition, the CDC GAN fails to take into account timing: it is trained on all
hit data from a beam MC simulation, which is mostly made up of hits from the
prompt products of the collision. Therefore, it would not be fit to generate a
sample of hits in the trigger time window, and another model trained on this
subset of hits would need to be trained.

% + Evaluating the quality of generated samples, in absolute terms or in physical
%   termsis difficult.


% Recommendations
% + Design algorithm not for efficiency, but for faithfulness.

\subsection{Mock data production}

\section{Atmospheric muon backgrounds}
