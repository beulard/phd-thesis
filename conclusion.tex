\chapter{Conclusion}

% Muon and LFV
Since the discovery of the muon in 1937, our understanding of its properties has
been steadily evolving. Today, although it fits nicely into the Standard Model
of particle physics, some questions are still standing, and recent tensions
observed between theory and experiment are suggesting that the SM does not give
a complete picture of the muon's underlying nature. 
% The context of this thesis
% lies in the current efforts to determine whether charged lepton flavour is
% violated. 
\hl{HERE}
% Bit weird. Maybe need a way to link the two sentences? (above and below)
In particular, the observation of neutrino-less muon to electron
conversion would yield clear evidence of charged lepton flavour violation,
because such a process ought to be suppressed beyond experimental sensitivities
according to theoretical constraints.

% COMET
The COMET Phase-I experiment will soon start its data acquisition run toward the
search for muon to electron conversion in aluminium. It is expected to be a
hundred times more sensitive to $\mu$--$e$ conversion than the previous best
measurement by SINDRUM II. To achieve this sensitivity, COMET Phase-I will
produce an intense pulsed muon beam directed toward an aluminium target to
create muonic aluminium atoms, from which conversion electrons may emerge. The
Cylindrical Detector, which surrounds the muon stopping target, is designed to
clearly identify conversion electrons while rejecting as many experimental
backgrounds as possible. 

% Software & sim
In order to cover the needs of the experiment in terms of simulation,
calibration, reconstruction, data formats and data analysis, the COMET
collaboration develops a comprehensive set of software utilities named ICEDUST.
The work presented throughout this thesis relies heavily on Monte Carlo (MC) data
produced by {\sc Geant4} simulations of Phase-I using ICEDUST.
% Here is a good place to kind of introduce the problem of producing large-scale
% datasets and estimating atmospheric backgrounds
MC simulations allow us to study the experiment's outcomes and to optimise its
design before assembly. However, they are computationally expensive and
typically only allow us to simulate a small fraction of the amount of data
expected to be collected. Additionally, traditional MC sampling is particularly
inefficient when the source of events is far from the detector system, as is the
case when considering cosmic ray-induced backgrounds. This thesis aims to
address these two limitations by partially circumventing the brute force MC
method.

% GAN
We investigate a novel approach to the mass-production of simulation data based
on Generative Adversarial Networks. We propose a neural network generator of
hits for the Cylindrical Drift Chamber, which can produce synthetic energy
deposits in the detector at a rate $10^6$ times higher than the ICEDUST
simulation. The machine learning model is trained on a sample of simulated hits
and learns from their features and relationships. The trained model allows us to
generate only a subset of the hits produced by MC simulation, but far more
efficiently than was previously achievable. This work and its future evolutions
will enable the production of mock datasets, which can be used by the
collaboration in anticipation of data acquisition runs.
%A future generative model might expand on this work by 

% BMC & background study
Additionally, we present a study of the pervasiveness of backgrounds caused by
cosmic ray-induced atmospheric muons in the COMET Phase-I muon to electron
conversion search. We use a backward Monte Carlo simulation method to
efficiently estimate the flux of atmospheric muons near the detector system. We
then perform a full analysis of the event count contributions from $\mu$--$e$
conversion, muon decay-in-orbit, and atmospheric muons. Our findings suggest
that an efficient rejection of atmospheric backgrounds by the Cosmic Ray Veto,
combined with particle-type identification by the Cylindrical Detector, are
necessary for the Phase-I search to succeed.


% Wrap up: what do my results mean for COMET!!!
