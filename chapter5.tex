\chapter{Backward Monte Carlo simulation for cosmic ray events}

\begin{markdown}

1. Intro to BMC: relation to adjoint MC (find refs), maybe diagram

2. Physics: how we use BMC to determine rates of cosmic events: equations for
flux, sampling PDF, BMC weights, topography map of J-PARC

3. Integration into ICEDUST: implementation of sampling methods, event loop, output to
oaRooTracker format
+ `https://gitlab.in2p3.fr/comet/ICEDUST_packages/-/merge_requests/534`
+ `https://gitlab.in2p3.fr/comet/ICEDUST_packages/-/wikis/Backward-Monte-Carlo-sampling-with-SimBackwardMC#rate-calculation`

3. Actual estimation of background rate in Phase-I, maybe just quoting V's
result


\end{markdown}

Cosmic ray-induced events represent a significant part of the expected
background in the COMET experiment, as discussed in
Section~\ref{sec:backgrounds}. In simulation studies, cosmic rays tend to be
expensive to simulate since they can fly in from any angle across a wide range
of energies. In order to avoid wasting computational time on events which are
unlikely to produce hits in the detector system, the flow of time in the Monte
Carlo simulation can be reversed. 

\section{Backward Monte Carlo simulation}
In typical forward Monte Carlo simulation, events are generated with a known
probability or rate and then propagated in time and space to determine their
outcomes. 

Backward Monte Carlo implies that events are generated close to the detector,
ensuring that they result in hits, and then propagated back to a boundary
where their flux is known in order to determine their rate.