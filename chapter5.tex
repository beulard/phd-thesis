\chapter{Backward Monte Carlo simulation for cosmic ray events}\label{ch:cosmics}

\begin{markdown}

1. Intro to BMC: relation to adjoint MC (find refs), maybe diagram

2. Physics: how we use BMC to determine rates of cosmic events: equations for
flux, sampling PDF, BMC weights, topography map of J-PARC

3. Integration into ICEDUST: implementation of sampling methods, event loop, output to
oaRooTracker format
+ `https://gitlab.in2p3.fr/comet/ICEDUST_packages/-/merge_requests/534`
+ `https://gitlab.in2p3.fr/comet/ICEDUST_packages/-/wikis/Backward-Monte-Carlo-sampling-with-SimBackwardMC#rate-calculation`

3. Actual estimation of background rate in Phase-I, maybe just quoting V's
result


\end{markdown}

% Intro: cosmics are important
Cosmic ray-induced events represent a significant part of the expected
background in the COMET experiment, as discussed in
Section~\ref{sec:backgrounds}. Cosmic muons irradiate the Earth at irregular
intervals and have a wide energy range, hence some fraction of them will be able
to mimic the conversion signal by entering the COMET detector system. 

Estimating the rate at which cosmic muons might produce signal-like events is
computationally expensive through standard Monte Carlo simulation methods. This
is due to the fact that the source of cosmic muons, i.e.\ the atmosphere, has a
much greater spatial extent than the active detector region. Hence, most cosmic
events generated from an atmospheric source will miss the detector completely
and not contribute in the estimation of the background rate, resulting in wasted
computation time.

A reverse (or adjoint) Monte Carlo simulation is one where the flow of particle
transport is reversed, i.e.\ events are generated in the sensitive detector
volume and propagated backward in time until they reach the source. This method
was initially used in 1967 to estimate gamma-ray and neutron radiation doses in
nuclear reactors~\cite{doi:10.13182/NSE68-A19235,doi:10.13182/NSE69-A19116}.
More recently, reverse MC has also been integrated into Geant4
simulations~\cite{DESORGHER2010247} for dosimetry in space, reversing the
electromagnetic physics of electrons, protons and ions.

In the case of muons, a backward transport method was developed by Niess et
al.~\cite{Niess_Barnoud_Carloganu_Menedeu_2018} in the context of muon
tomography, and it was later adapted to the COMET experiment in order to refine
estimations of the background rate from cosmic rays. This chapter discusses the
backward MC method for cosmic muons, how it is applied to COMET and the
resulting rate estimations in COMET Phase-I.


\section{Backward Monte Carlo transport}
In standard Monte Carlo simulations, events are generated by a source and
propagated forward in time according to their equations of motion and
interaction cross-sections within any surrounding medium. In a situation where
the source is large compared to the sensitive detector volume, as illustrated in
Figure~\ref{fig:bmc_configuration}, events have a high probability to miss the
detector and contribute nothing in the desired computation.

In backward MC, events are generated according to an arbitrary probability
density function (PDF) around the detector volume. The overall sampling PDF is
typically the composite of a position, direction and energy sampling
distributions, i.e.\ 
$$
p(\bm{x}, \bm{p}, E) = p(\bm{x}) \times p(\bm{p}) \times p(E),
$$
where $p$ denotes a PDF whose integral is 1, $\bm{x}$ denotes position, $\bm{p}$
direction and $E$ energy. 

\sepfootnotecontent{A}{See Reference~\cite{DESORGHER2010247} for a thorough
discussion of adjoint cross-sections, weights, weight corrections and source
normalisation.}

Each event is propagated backward using adjoint transport and interaction
kernels such that the probability of the event can eventually be determined. At
each step of the propagation, a multiplicative weight is computed from the
adjoint interaction cross-sections in the local medium. When the particle
reaches the source, the directional flux at the source is used as a
normalisation factor to yield the final event weight\sepfootnote{A}. Given a
significant enough sample of reverse events, the weights can be summed to
obtain the absolute flux or rate of particles hitting the detector.


% Interesting paragraph from desorgher:
%  The Monte Carlo sampling of a big enough number of reverse tracks is
%  equivalent to the integration of the weight W over all independent variable
%  (E1,O1,Sd,x1,E0, ...) summed over all type of reverse reactions, atomic
%  elements, adjoint primaries, and adjoint secondaries. In this integration the
%  cases where more than one reaction and no reaction occur are also considered
%  and only the tracks reaching the external source are accounted for. By this
%  way the same answer is obtained as in the forward case.

\begin{figure}
    \centering
    %\includegraphics[width=0.6\textwidth]{chapter5/qwe.png}
    \caption{BMC configuration.}
    \label{fig:bmc_configuration}
\end{figure}

% How does one estimate the rate of cosmics from the sampling PDF and BMC
% weights?
% Describe procedure / maths?

% We backward propagate the same event many times in order to get a reasonable
% rate estimate. Perhaps mention that.

\section{Cosmic muon rate estimation}
% Here, talk about the COMET setup: 
% + We want to estimate the rate of background events caused by 
%   muons+secondaries hitting the detector
% + Thus, we sample muons+/- around the envelope of the CRV, according to [spell
%   out PDF]
% + We apply backward MC up to an atmospheric muon flux model at altitude X,
%   tabulated from CORSIKA etc.
% + We can first estimate the cosmic muon rate on each surface of the CRV
%   Show plots etc.
% + From the events generated at the CRV envelope, it is also possible to
%   propagate forward to determine which events might produce signal-like
%   tracks.
% + Once this is done, we use the results from backward propagation to estimate
%   rate of background-inducing cosmic muons, qed.


% Do we talk about the PUMAS, GOUPIL setup?
% ICEDUST integration?