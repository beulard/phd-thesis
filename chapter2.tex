\chapter{The COMET experiment}\label{chapter2}

\begin{markdown}
---

- Description of the COMET experiment's goal, design with nice illustrations
    + *Reference next chapter for geometry renderings*
    + Signal and background:
        + mu-e conv signal description
        + List of background sources
+ CyDet:
    + For simulation section, need to explain how CDC and CTH work, and how combined they enable mu--e conv measurement
    - Detailed description of the CDC, which is crucial for the GAN section.
+ Phase alpha

- References: TDR, SINDRUM-II, 

---
\end{markdown}

The COMET (COherent Muon-to-Electron Transition) project \hl{project?} is a muon-beam experiment aiming to observe the muon-to-electron conversion process, or at least to constrain its branching ratio to an unprecedented upper limit.
Currently being built at the J-PARC facility in Tokai, Japan, its physics program will be run in a two-stage approach, Phase-I and Phase-II. These two phases are expected to improve our sensitivity to the conversion signal by factors of 100 and \num{10000}, respectively, with respect to the current world-leading measurement conducted at the SINDRUM II experiment~\cite{Bertl:2006up}.



