\chapter{The COMET Experiment}\label{chapter2}

\begin{markdown}
---

- Description of the COMET experiment's goal, design with nice illustrations
    + *Reference next chapter for geometry renderings*
    + Signal and background:
        + mu-e conv signal description
        + **List of background sources**
+ CyDet:
    + For simulation section, need to explain how CDC and CTH work, and how combined they enable mu--e conv measurement
    - Detailed description of the CDC, which is crucial for the GAN section.
     - Stereo angles
+ Phase alpha?

- References: TDR, SINDRUM II, 

---

+ Requirements: high sensitivity to signal, efficient rejection of backgrounds
 + -> Need intense muon beam, pulsed, and detector design must avoid backgrounds
+ TIMING of signal (muon lifetime)
+ Proton beam energy: why 8 Gev -> antiproton production
+ Intensity: beam current, beam power, POTs per second
+ Send backward-going secondaries to detector, discard the main part of secondaries (forward-going)
+ Curved solenoid + dipole field (by tilting coils, see Krikler)
+ Stopping target -> why Al
+ Bunch structure, extinction
+ Phase-I detectors: StrECAL and CyDet

---
\end{markdown}

% Description and goals
COMET (COherent Muon-to-Electron Transition) is a future muon-beam experiment
designed to search for the muon-to-electron conversion process. It is currently
under construction at the Japan Proton Accelerator Research Complex (J-PARC)
facility in Tokai, Japan. The goal of COMET is to be \numprint{10000} times more
sensitive to $\mu$--$e$ conversion than the current world-leading limit set by
the SINDRUM II experiment~\cite{Bertl:2006up}. 

% Requirements
In order to reach these goals, the COMET experiment is designed with strict
requirements defined to make the conversion signal as clear as possible, while
efficiently rejecting background events. The most essential requirements are:
\begin{itemize}
    \item An intense muon source to probe the rarest of events,
    \item A pulsed beam such that timing information can be used to reject backgrounds,
    \item Strict selection of the beam particles' charge and momentum prior to reaching
    the detector.
\end{itemize}
These requirements and the design choices that were made to address them are
described in more detail in Section~\ref{Phase-I_requirements}.
% Later, go back to requirements and explain in more detail and then tell what
% the design choices were to address these requirements.

% Strategy
COMET is planned to run in a staged approach such that the properties of the
newly built beam can be finely understood before making the measurement. COMET
Phase-I aims to reach a single-event sensitivity\footnote{\hl{Definition?}} to $\mu$--$e$ conversion of
$3\times 10^{-15}$, a factor-100 improvement over SINDRUM II.

\section{Muon production}
Searching for $\mu$--$e$ conversion requires an intense source of muons. The
COMET experiment relies on the J-PARC accelerator facility to provide protons.
The proton beam hits a static graphite target to produce pions, which then decay
to muons. In order for the collision to produce a reasonable number of pions
while avoiding anti-proton production\footnote{
Anti-protons have the same charge as pions but travel more slowly for a given
momentum. Hence, they can produce delayed secondaries which hit the detector
system, which is a source of background as discussed in
Section~\ref{sec:delayed_backgrounds}. A beam energy of \SI{8}{\GeV} is the
lower limit above which anti-proton production becomes kinematically allowed.
}, the protons in the beam have a kinetic energy of \SI{8}{\GeV}. 


\section{Experimental backgrounds}\label{sec:backgrounds}
The COMET experiment is specifically designed to be sensitive to a $\mu$--$e$
conversion signal, i.e.\ to eliminate as many sources of background as possible
while making sure to notice the process if it does occur. 

Observing $\mu$--$e$ conversion requires the production of an intense source of
muons which must be bound to atomic nuclei. COMET uses a proton beam from the
J-PARC main ring.


The COMET experiment will produce a
muon beam and direct it toward a static aluminium target to slow down and stop
the muons. As discussed in Section~\ref{sec:sm_backgrounds}, a bound muon can
undergo decay in orbit (DIO) or nuclear capture. 

\section{Design requirements}

The design of the COMET experiment is heavily angled toward suppressing
backgrounds to the $\mu$--$e$ conversion signal. Understanding the background
sources and their properties is key in making sense of the experimental design.

\section{The COMET beam}\label{sec:COMET_beam}


% The choice of target material directly influences how frequently high-energy
% electrons will be produced by these processes, and hence contributes in
% setting the maximum experimental sensitivity.

% This should go into the COMET section as these are specific to COMET and/or
% specifically addressed by the COMET design.
% Background events can be classified into three categories: intrinsic,
% beam-related, and cosmic ray-induced.

% Beam-related backgrounds come from impurities in the high-intensity muon beam.
% The specifics of the COMET beam and how these backgrounds are alleviated is
% discussed in Section~\ref{sec:COMET_beam}.

% Cosmic rays typically produce muons with a wide range of energies in the
% atmosphere which 