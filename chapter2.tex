\chapter{The COMET experiment}\label{chapter2}

\begin{markdown}
---

- Description of the COMET experiment's goal, design with nice illustrations
    + *Reference next chapter for geometry renderings*
    + Signal and background:
        + mu-e conv signal description
        + **List of background sources**
+ CyDet:
    + For simulation section, need to explain how CDC and CTH work, and how combined they enable mu--e conv measurement
    - Detailed description of the CDC, which is crucial for the GAN section.
     - Stereo angles
+ Phase alpha

- References: TDR, SINDRUM II, 

---
\end{markdown}

COMET (COherent Muon-to-Electron Transition) is a muon-beam experiment aiming to
observe the muon-to-electron conversion process, or at least to constrain its
branching ratio to an unprecedented upper limit. Currently being built at the
J-PARC facility in Tokai, Japan, its physics program will be run in a two-stage
approach, Phase-I and Phase-II. These two phases are expected to improve our
sensitivity to the conversion signal by factors of 100 and \num{10000},
respectively, with respect to the current world-leading measurement conducted at
the SINDRUM II experiment~\cite{Bertl:2006up}.




The design of the COMET experiment is heavily angled toward suppressing
backgrounds to the $\mu$--$e$ conversion signal. Understanding the background
sources and their properties is key in making sense of the experimental design.

\section{The COMET beam}\label{sec:COMET_beam}

\section{Experimental backgrounds}\label{sec:backgrounds}

% The choice of target material directly influences how frequently high-energy
% electrons will be produced by these processes, and hence contributes in
% setting the maximum experimental sensitivity.

% This should go into the COMET section as these are specific to COMET and/or
% specifically addressed by the COMET design.
% Background events can be classified into three categories: intrinsic,
% beam-related, and cosmic ray-induced.

% Beam-related backgrounds come from impurities in the high-intensity muon beam.
% The specifics of the COMET beam and how these backgrounds are alleviated is
% discussed in Section~\ref{sec:COMET_beam}.

% Cosmic rays typically produce muons with a wide range of energies in the
% atmosphere which 