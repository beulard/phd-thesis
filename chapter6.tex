\chapter{COMET Phase-I sensitivity and backgrounds}

With the framework to simulate backgrounds from atmospheric muons in place, we
can now combine the results into a complete sensitivity and background
simulation study for COMET Phase-I. In this chapter, we discuss our method to
produce a signal sample and a muon decay-in-orbit sample, and present our
resulting expectations of the performance of Phase-I.

\section{Simulation samples}
Three simulation samples are produced in this study. A $\mu$--$e$ conversion
sample, a DIO sample and an atmospheric muon sample (identical to that of
Chapter~\ref{ch:cosmics}). 

\subsection{$\mu$--$e$ conversion signal}
The signal sample is fairly straightforward to produce. We initially generate
primary electrons with energy $E=\SI{104.97}{\MeV}$ uniformly inside the
stopping target disks. Their initial direction is isotropically distributed, as
would be the case in the conversion process. The uniform position distribution
in each disk is not realistic, hence the events will be weighted according to a
realistic beam simulation where the position of muons stopping in the disks was
determined. In total, $N_\mathrm{signal} = 2\times 10^6$ events are simulated
for the signal sample.

\hl{mention mc5 for position distribution}

% After applying geom cut of CTH trigger, how many remain?

\begin{figure}
    \centering
    
    \caption{Conversion signal event in the CyDet system.}
    \label{fig:signal_in_cydet}
\end{figure}


\subsection{Muon decay-in-orbit background}
\begin{figure}
    \centering
    
    \caption{DIO-induced background with momentum  $p=\SI{1}{\MeV/\clight}$.}
    \label{fig:muon_dio_in_cydet}
\end{figure}

\subsection{Atmospheric muon background}
\begin{figure}
    \centering
    
    \caption{Cosmic ray-induced background with momentum $p=\SI{1}{\MeV/\clight}$.}
    \label{fig:cosmic_bg_in_cydet}
\end{figure}

The time distribution of signal, DIO and atmospheric events is not realistically
simulated in this study. Instead, we apply an average timing window efficiency
weight to all events in order to estimate the total count integrated over
data-acquisition time.

\section{Single event sensitivity}
The single event sensitivity (SES) is defined as the value of the $\mu$--$e$
conversion branching ratio required for the experiment to observe one signal
event. It can be expressed in terms of the experimental acceptance $A_{\mu-e}$ and the
total number of muons stopped in the stopping target $N_\mu$:
\begin{equation}
    \mathrm{SES} = \frac{1}{N_\mu\,A_{\mu-e}\,\mathcal{B}_\mathrm{capture}\,f_\mathrm{coherent}},
\end{equation}
where $\mathcal{B}_\mathrm{capture} = 0.61$ is the branching ratio of nuclear
muon capture and $f_\mathrm{coherent} = 0.9$ is the fraction of conversions
expected to leave the nucleus in its ground state.

% Is it worth ELI5 here?
% The net signal acceptance $A_{\mu-e}$ is the product of various factors stemming
% from the experiment's design and inefficiencies in the processing,
% reconstruction and selection of candidate events. 

% How do we approach the fact that CTH geom has changed so our geom acceptance
% is lower?
In this study, the CTH trigger mechanism was simulated by finding events
involving a fourfold coincidence. The fraction of signal events which induce a
trigger is defined as the geometrical acceptance $A_\mathrm{geom}$. In our
simulation setup, the CTH contains 64 counters per layer, two layers
in each module, one module at the upstream end of the CDC and the other
downstream (256 counters in total). Our estimated geometrical acceptance is
$$
A_\mathrm{geom} \equiv \frac{\text{fourfold coincidences}}{N_\mathrm{signal}} = \SI{21}{\percent}.
$$
In comparison, the COMET Phase-I TDR~\cite{the_comet_collaboration_comet_2020}
cites $A_\mathrm{geom} = \SI{26}{\percent}$ with the previous design of the CTH,
which had 48 counters per layer. This reduced acceptance worsens the sensitivity
of the experiment slightly. In this study, none of the other experimental
aspects that contribute to the net acceptance were changed, hence we use the
other factors of $A_{\mu-e}$ from the TDR, and only replace the value of
$A_\mathrm{geom}$ with the newly estimated one. Values of the acceptance
coefficients used in our study are shown in Table~\ref{tab:acceptance}. The net
signal acceptance using these factors is $A_{\mu-e} = 0.032$.

The number of muons stopped in the stopping target, $N_\mu$, is a function of
the COMET beam power, the target material, the design of the
transport system, and the total data-acquisition time. 
The target material sets the yield of
backward-going pions which will decay to muons of the right momentum to come at
rest in the muon stopping target. This, combined with the transport system,
determines the yield of stopped muons per proton collision $R_{\mu/p} \equiv
\frac{\text{muons stopped}}{\text{proton on target}}$. Here, we use the
MC5 dataset, a sample of $10^9$ POT collisions, to estimate $R_{\mu/p}=4.86
\times 10^{-4}$.
The beam power and data-acquisition time together determine the total number of proton
collisions that will occur. COMET Phase-I requires $N_\mathrm{POT} = 3.15 \times
10^{19}$ collisions in order to reach its sensitivity goal, corresponding to a
data-acquisition time of $T=146$ days.
These figures allow us to estimate the total number of stopped muons
$$N_\mu = R_{\mu/p} \times N_\mathrm{POT} = 1.56\times 10^{16},$$
which can be substituted into the SES formula:
\begin{equation}\label{eq:my_ses}
\mathrm{SES}
=\frac{1}{N_\mu\,A_{\mu-e}\,\mathcal{B}_\mathrm{capture}\,f_\mathrm{coherent}}
= 3.81\times10^{-15}.
\end{equation}
This value is slightly worse than the TDR estimation,
$\mathrm{SES}_\mathrm{TDR}=3\times 10^{-15}$, because of our lower geometrical
acceptance by the new CTH design.

% \begin{table}
%     \centering\begin{tabular}{l|ccc}
%         \toprule
%         a&b&c&d\\\midrule
%         a&b&c&d\\
%         \bottomrule
%     \end{tabular}
%     \caption{ Acceptance coefficients used to estimate the single event
%     sensitivity in Equation~\ref{eq:my_ses}. These also apply to the
%     decay-in-orbit background rate estimation. }
%     \label{tab:ses_efficiency_factors}
% \end{table}


\section{Momentum spectrum}

\section{Discussion}

\subsection{Atmospheric $\mu^+$ backgrounds}

\subsection{CRV efficiency}

\subsection{Particle identification}
