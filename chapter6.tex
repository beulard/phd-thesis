\chapter{COMET Phase-I sensitivity and backgrounds}

With the framework to simulate backgrounds from atmospheric muons in place, we
can now combine the results into a complete sensitivity and background
simulation study for COMET Phase-I. In this chapter, we discuss our simulation
samples consisting of signal, decay in orbit (DIO), and atmospheric events, and
present our resulting expectations of the performance of Phase-I.

\section{Simulation setup}
Three simulation samples are produced for this study. A $\mu$--$e$ conversion
sample, a DIO sample and an atmospheric muon sample. All three are simulated in
the geometrical world shown in Figure~\ref{fig:bmc_geometry}. This geometry
differs from the older TDR design~\cite{the_comet_collaboration_comet_2020} by
the number of counters in the CTH. Here, each layer of the CTH has 64 counters
instead of the original 48, which mainly affects the detector's geometrical
acceptance. In addition, the outer layer of the CTH is composed of plastic
scintillator counters rather than the Cherenkov counters, as was originally
planned. As we will see, this has an important effect on the detector's ability
to discriminate atmospheric anti-muons from conversion electrons.

\subsection{\texorpdfstring{$\mu$--$e$}{Muon to electron} conversion} 

\subsubsection{Sample}

The signal sample is the most straightforward to produce. We initially generate
primary electrons with energy $E=\SI{104.97}{\MeV}$ uniformly inside the
stopping target disks. Their direction is isotropically distributed, as would be
the case in the conversion process. A uniform position distribution in each disk
is not realistic, because the actual distribution depends on where in the
stopping target the muons in the beam become bound. To account for this, the
events are weighted according to the stopping positions of muons recorded in the
MC5 simulation. The weighting factor is determined from the relative probability
for a muon to stop at the sampled position, which is estimated by histogramming
the stopping positions in each disk.
Figure~\ref{fig:stopping_position_reweighting} shows the sampled uniform
position distribution and the resulting distribution after weighting. In total,
$N_\mathrm{signal} = 2\times 10^6$ events are simulated to compose the signal
sample.

% After applying geom cut of CTH trigger, how many remain?

\begin{figure}
    \centering
    \begin{subfigure}{0.46\textwidth}
        \centering
        \caption{Pre-weighting.}
    \end{subfigure}
    \hfill
    \begin{subfigure}{0.46\textwidth}
        \centering
        \caption{Post-weighting.}
    \end{subfigure}
    \caption{ Initial position distribution of signal electrons before and after
        weighting them by the likelihood of a muon being bound in each bin. }
    \label{fig:stopping_position_reweighting}
\end{figure}

\begin{figure}
    \centering
    
    \caption{Conversion signal event in the CyDet system.}
    \label{fig:signal_in_cydet}
\end{figure}

\subsubsection{Selection}
Signal events are selected based on detector acceptance criteria. We first require
fourfold coincidence in the CTH. The fraction of events remaining
defines the geometrical acceptance 
$$
A_\mathrm{geom} \equiv  \frac{\text{fourfold coincidences}}{N_\mathrm{signal}}.
$$
In our simulation setup with the 64-counters-per-layer CTH, our estimated
geometrical acceptance is $A_\mathrm{geom} = \SI{21}{\percent}$. In comparison,
the TDR cites $A_\mathrm{geom} = \SI{26}{\percent}$ with the previous design of
the CTH. 
\hl{Figure~X shows a conversion event which passes the
fourfold coincidence selection criterion.??}

% Reconstruction
We do not fully simulate the reconstruction of signal electron trajectories in
the CDC. In order to approximate the effect of reconstruction uncertainties, a
smearing is applied to the true momentum of each track. The reconstructed
momentum is estimated as $p_r = p_t + x$, where $p_t$ is the true momentum of
the electron as it enters the CDC, $x \sim \mathcal{N}(0, \sigma)$, and $\sigma
= \SI{200}{\keV/\clight}$ is the expected momentum resolution of the CDC.

% Other acceptance factors (not simulated)
Other acceptance criteria, such as the trigger timing window and reconstruction
quality cuts, are not properly simulated in this study. Instead, we apply
efficiency factors for each of them based on their TDR estimates.
\hl{Reference a table here?}


\subsection{Muon decay in orbit}
\subsubsection{Sample}
The DIO sample is similar to the signal sample in that the initial position
of signal and DIO electrons is identically distributed. Hence, we also sample
uniformly in the stopping target disks and then weight the events according to
the MC5 stopping position distribution. Similarly, the direction of DIO
electrons is sampled isotropically. The energy distribution is thus the only
difference between the two samples. 

\begin{figure}
    \centering
    
    \caption{Energy spectrum of DIO electrons from muonic aluminium atoms.}
    \label{fig:czarnecki_spectrum}
\end{figure}

The theoretical energy spectrum of DIO electrons produced in aluminium muonic
atoms was determined by Czarnecki et al.~\cite{czarnecki} and is shown in
Figure~\ref{fig:czarnecki_spectrum}. Their calculated spectrum is used to
determine an energy weighting factor for simulated DIO events. In a similar way
to the position weighting procedure, DIO events are first generated with a
uniform energy distribution, and then weighted according to the relative
likelihood for a DIO electron to have the sampled energy.
Figure~\ref{fig:dio_energy_reweighting} shows the weighted energy distribution
compared to the theoretical energy spectrum. \hl{FIGURE and then adjust text}.
The DIO sample is composed of $N_\mathrm{DIO} = 10^7$ events in total.

\begin{figure}
    \centering
    
    \caption{DIO-induced background with momentum  $p=\SI{1}{\MeV/\clight}$.}
    \label{fig:muon_dio_in_cydet}
\end{figure}

\subsubsection{Selection}
The selection of DIO events is identical to the selection of conversion events.
We require a fourfold coincidence in the CTH, and then apply a Gaussian smear to
the true momentum of the electron to simulate track reconstruction. The same
efficiency factors as for signal events are applied.

\subsection{Atmospheric muons}

\subsubsection{Sample}

Atmospheric muon events are simulated as explained in
Section~\ref{sec:cosmic_event_sampling}. Two samples are produced, one on the
entire surface of the CRV and one more densely concentrated on its upstream and
downstream openings. Because of how rarely a sampled atmospheric event produces
signal-like features, the atmospheric dataset is by far the largest of the
three, with $1 \times 10^9$ events generated on the envelope and $1.4 \times
10^9$ on the openings.



\begin{figure}
    \centering
    
    \caption{Cosmic ray-induced background with momentum $p=\SI{1}{\MeV/\clight}$.}
    \label{fig:cosmic_bg_in_cydet}
\end{figure}


\hl{TODO remove below}

The time distribution of signal, DIO and atmospheric events is not realistically
simulated in this study. Instead, we apply an average timing window efficiency
factor to all events in order to estimate the total count integrated over
data-acquisition time.

The time efficiency factor is not the same for signal, DIO and cosmics,
because they don't have the same time distribution. For cosmics, the time
distribution is ~uniform, so the factor should just be live time / (live +
dead time). For signal and DIO, it is dependent upon the time distribution of
muon bound in the tgt.

\subsubsection{Selection}
\hl{TODO}


\subsection{Sample weighting}
\sepfootnotecontent{fn:conv_br_norm}{The nuclear muon capture branching ratio appears in this
expression because the branching ratio of $\mu$--$e$ conversion is
conventionally normalised to $\mathcal{B}_\mathrm{capture}$.}

Because they originate from different processes, the three MC samples must be
weighted differently in order to determine the absolute contribution from each.
The conversion and DIO processes both originate from the muons bound in the
stopping target, hence we can express the total number of expected conversion
and DIO electrons in terms of the total number of stopped muons $N_\mu$.
The number of signal electrons, as a function of the conversion branching
ratio $\mathcal{B}_\mathrm{conversion}$, is:
$$
N_{e^-}^\mathrm{conversion} = 
N_\mu \, \mathcal{B}_\mathrm{conversion} \, 
\mathcal{B}_\mathrm{capture} \, f_\mathrm{coherent},
$$
where $\mathcal{B}_\mathrm{capture} = 0.61$ is the branching ratio of nuclear
muon capture\sepfootnote{fn:conv_br_norm} and $f_\mathrm{coherent}=0.9$ is the
fraction of conversions that are coherent. Similarly, the total number of DIO
electrons can be expressed as:
$$
N_{e^-}^\mathrm{DIO} = N_\mu \, \mathcal{B}_\mathrm{DIO},
$$
where $\mathcal{B}_\mathrm{DIO} = 1 - \mathcal{B}_\mathrm{capture} -
\mathcal{B}_\mathrm{conversion} \approx 0.39$ is the branching ratio of DIO.


In the case of atmospheric muons, the backward MC procedure yields an absolute
rate. This rate is multiplied by the total data acquisition (DAQ) time of the
experiment to obtain the expected background count:
$$
N_\mathrm{atmospheric} = R \times T_\mathrm{DAQ},
$$
where $R$ is the total atmospheric background rate, and we set
$T_\mathrm{DAQ}=146$ days, corresponding to $N_\mu = 1.5\times 10^{19}$, in this
study. 

\subsubsection{Trigger time window efficiencies}
In order to take into account the trigger timing window, we assume that
atmospheric muons irradiate the detector with a uniform time distribution.
Hence, the time window efficiency factor is simply the average fraction of
time when the trigger is active:
$$
\epsilon_\text{time window}^\mathrm{atmospheric} =
\frac{8}{9}\,\frac{1170 - 700}{1170} = \SI{36}{\percent}
$$
assuming the trigger window is between 700 and \SI{1170}{\ns}, and where the
factor $\frac{8}{9}$ arises from the bunch structure of the J-PARC main ring. In
contrast, the efficiency factor for conversion and DIO electrons
$\epsilon_\text{time window}^\text{conversion|DIO} \approx \SI{30}{\percent}$ is
smaller because the time distribution of bound muons peaks around \SI{300}{\ns},
before the trigger becomes active (see Figure~\ref{fig:timing_distributions}).



\section{Single event sensitivity}
The single event sensitivity (SES) is defined as the value of the $\mu$--$e$
conversion branching ratio required for the experiment to observe one signal
event. It can be expressed in terms of the experimental acceptance $A_{\mu-e}$ and the
total number of muons stopped in the stopping target $N_\mu$:
\begin{equation}
    \mathrm{SES} = \frac{1}{N_\mu\,A_{\mu-e}\,\mathcal{B}_\mathrm{capture}\,f_\mathrm{coherent}},
\end{equation}
where $\mathcal{B}_\mathrm{capture} = 0.61$ is the branching ratio of nuclear
muon capture and $f_\mathrm{coherent} = 0.9$ is the fraction of conversions
expected to leave the nucleus in its ground state.

% Is it worth ELI5 here?
% The net signal acceptance $A_{\mu-e}$ is the product of various factors stemming
% from the experiment's design and inefficiencies in the processing,
% reconstruction and selection of candidate events. 

% How do we approach the fact that CTH geom has changed so our geom acceptance
% is lower?

\hl{ v We've already mentioned trigger selection and acceptance in the Selection
    subsec of every sample. Cut.}
    
In this study, the CTH trigger mechanism was simulated by finding events
involving a fourfold coincidence. The fraction of signal events which induce a
trigger is defined as the geometrical acceptance $A_\mathrm{geom}$. In our
simulation setup, the CTH contains 64 counters per layer, two layers
in each module, one module at the upstream end of the CDC and the other
downstream (256 counters in total). Our estimated geometrical acceptance is
$$
A_\mathrm{geom} \equiv \frac{\text{fourfold coincidences}}{N_\mathrm{signal}} = \SI{21}{\percent}.
$$
In comparison, the COMET Phase-I TDR~\cite{the_comet_collaboration_comet_2020}
cites $A_\mathrm{geom} = \SI{26}{\percent}$ with the previous design of the CTH,
which had 48 counters per layer. This reduced acceptance worsens the sensitivity
of the experiment slightly. In this study, none of the other experimental
aspects that contribute to the net acceptance were changed, hence we use the
other factors of $A_{\mu-e}$ from the TDR, and only replace the value of
$A_\mathrm{geom}$ with the newly estimated one. Values of the acceptance
coefficients used in our study are shown in Table~\ref{tab:acceptance}. The net
signal acceptance using these factors is $A_{\mu-e} = 0.032$.

The number of muons stopped in the stopping target, $N_\mu$, is a function of
the COMET beam power, the target material, the design of the
transport system, and the total data-acquisition time. 
The target material sets the yield of
backward-going pions which will decay to muons of the right momentum to come at
rest in the muon stopping target. This, combined with the transport system,
determines the yield of stopped muons per proton collision $R_{\mu/p} \equiv
\frac{\text{muons stopped}}{\text{proton on target}}$. Here, we use the
MC5 dataset, a sample of $10^9$ POT collisions, to estimate $R_{\mu/p}=4.86
\times 10^{-4}$.
The beam power and data-acquisition time together determine the total number of proton
collisions that will occur. COMET Phase-I requires $N_\mathrm{POT} = 3.15 \times
10^{19}$ collisions in order to reach its sensitivity goal, corresponding to a
data-acquisition time of $T=146$ days.
These figures allow us to estimate the total number of stopped muons
$$N_\mu = R_{\mu/p} \times N_\mathrm{POT} = 1.56\times 10^{16},$$
which can be substituted into the SES formula:
\begin{equation}\label{eq:my_ses}
\mathrm{SES}
=\frac{1}{N_\mu\,A_{\mu-e}\,\mathcal{B}_\mathrm{capture}\,f_\mathrm{coherent}}
= 3.81\times10^{-15}.
\end{equation}
This value is slightly worse than the TDR estimation,
$\mathrm{SES}_\mathrm{TDR}=3\times 10^{-15}$, because of our lower geometrical
acceptance when using the new CTH design.

% \begin{table}
%     \centering\begin{tabular}{l|ccc}
%         \toprule
%         a&b&c&d\\\midrule
%         a&b&c&d\\
%         \bottomrule
%     \end{tabular}
%     \caption{ Acceptance coefficients used to estimate the single event
%     sensitivity in Equation~\ref{eq:my_ses}. These also apply to the
%     decay-in-orbit background rate estimation. }
%     \label{tab:ses_efficiency_factors}
% \end{table}


\section{Momentum spectra}
Because the signal has a clear mono-energetic signature, a precise measurement
of the momentum of candidate tracks helps to discriminate signal from
backgrounds. Figure~\ref{fig:log_spectrum} shows the momentum spectrum
for candidate conversion, DIO and atmospheric events.

\begin{figure}
    \centering
    
    \caption{Log spectrum.}
    \label{fig:log_spectrum}
\end{figure}

\section{Discussion}

\subsection{Atmospheric $\mu^+$ backgrounds}

\subsection{CRV efficiency}

\subsection{Particle identification}
