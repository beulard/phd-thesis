\chapter{Lepton Flavour in the Standard Model and Beyond}\label{chapter1}

% Crazy superlatives about the SM
The Standard Model (SM) of particle physics is possibly the most successful
mathematical model of physical phenomena so far. It provides an accurate
description of almost all observable interactions between known elementary
particles. It yields predictions for what nature does and does not allow, and
enables physicists to examine and test the fundamental laws of the universe. It
provides a rigid framework to make theoretical predictions, from which any
measured deviations can then be interpreted as new physics.

% Bla bla bla
Charged lepton flavour conservation, which stems from an accidental symmetry of
the SM, is currently under strict scrutiny by various experiments around the
world. This strong interest was in part driven by the observation of neutrino
oscillations, which has proven that the current formulation of the SM cannot
fully describe all fundamental interactions of nature. If charged lepton flavour
violation is observed, not only will it add to the current evidence that the SM
is incomplete, but it will help guide us toward the next theory of particle
physics beyond the Standard Model.

The most sensitive probe to search for CLFV is the muon. This chapter describes
how the muon has come from being a mysterious particle showering Earth from
outer space to becoming a tool in high energy physics experiments used to
explore the frontier of our knowledge of nature.


\section{Discovery of the muon}
The first traces of muons were observed around 1937 by three experiments
investigating the nature of cosmic ray-induced particle
showers~\cite{PhysRev.51.884, PhysRev.52.1198, PhysRev.52.1003}. One of them was
able to estimate the mass of the discovered particle at hundreds of times that of the
electron: around the same as the strong-force-carrying meson predicted by Yukawa
in 1935~\cite{10.1143/PTPS.1.1}. Hence, the muon and Yukawa's particle were
originally believed to be one and the same particle. It was only a decade later,
when the meson (now called $\pi$, for primary) was observed decaying into a
muon, that the two particles were completely disambiguated~\cite{LATTES1947}.

It was the fact that the muon appeared as nothing but a heavy electron which
prompted Rabi to ask ``who ordered that?'' in response to its discovery. In
fact, even the Standard Model of today cannot give a satisfactory answer to this
question, since the SM does not motivate the existence of three generations of
elementary particles. As far as the SM can explain, there is no fundamental reason
for the existence of distinct flavours.

\section{The muon in the Standard Model}
% 
The SM identifies the muon as the second-generation charged lepton, meaning it
is a fermion with identical properties, aside from flavour and mass, as the electron
and tau lepton. A muon in a vacuum can only decay through the weak force. The
diagram for muon decay, $\mu^- \rightarrow e^- +  \nu_\mu + \overline{\nu}_e$,
is shown in Fig.~\ref{fig:weak_decay}.


\begin{figure}
    \centering
    \feynmandiagram [layered layout, horizontal=a to b] {
        a [particle=\(\mu^{-}\)] -- [fermion] b -- [fermion] f1 [particle=\(\nu_{\mu}\)],
        b -- [boson, edge label'=\(W^{-}\)] c,
        c -- [anti fermion] f2 [particle=\(\overline \nu_{e}\)],
        c -- [fermion] f3 [particle=\(e^{-}\)],
    };
    \caption{Feynman diagram for the weak decay of the muon.}
    \label{fig:weak_decay}
\end{figure}

In the SM Lagrangian with massless neutrinos, none of the terms which involve
leptons allow for flavour violation. The Lagrangian is invariant under
transformations of the ${U(1)_e \times U(1)_\mu \times U(1)_\tau}$ group.
Consequently, each lepton family ($e$, $\mu$, $\tau$) has its own conserved
number. In theory, this completely prevents a charged lepton from changing
flavour without neutrinos being involved to balance the process. 

These conservation laws do not correspond to a fundamental symmetry of nature;
they are merely an accidental feature of the SM Lagrangian and have, so far,
been observed to hold experimentally. For instance, the process ${\mu
\rightarrow e + \gamma}$, in principle allowed by kinematics, has never been
observed and the current upper limit on its branching ratio was set by the MEG
experiment at $10^{-13}$~\cite{mori2016final}.

The observation of neutrino oscillations~\cite{PhysRevLett.81.1562} means that
the three accidental flavour symmetries are not exact. This strongly motivates a
search for flavour violation among the charged leptons as well, as any evidence
for it would yield additional hints about the theory lying beyond the Standard
Model (BSM).

\begin{figure}
\centering
\begin{subfigure}[t]{0.45\textwidth}
    \centering
    \begin{tikzpicture}
    \begin{feynman}
    \vertex (a) {\(\mu\)};
    \vertex [right=1.5cm of a] (b);
    \vertex [right=1.8cm of b] (c);
    \vertex [right=1.5cm of c] (d) {\(e\)};
    
    \vertex at ($(b)!0.5!(c) + (0, 0.9cm)$) (n);
    \vertex at ($(b)!0.5!(c) + (0, -0.9cm)$) (w);
    \vertex [above=0.1cm of w] (wn) {\(W\)};
    \vertex [below=1.5cm of d] (g) {\(\gamma\)};
    
    \diagram* {
    (a) -- [fermion] (b)
    -- [fermion, quarter left, edge label=\(\nu_\mu\)] (n)
    -- [fermion, quarter left, edge label=\(\nu_e\)] (c),
    (b) -- [boson, quarter right] (w)
    -- [boson, quarter right] (c),
    (c) -- [fermion] (d),
    (w) -- [boson] (g),
    };
    
    \draw (n) -- (n) node {\(\times\)};
    
    \end{feynman}
    \end{tikzpicture}
    \caption{
        $\mu \rightarrow e + \gamma$
    }
    \label{fig:mu_e_nu_osc}
\end{subfigure}
\begin{subfigure}[t]{0.45\textwidth}
    \centering
    \begin{tikzpicture}
    \begin{feynman}
    \vertex (m1) {\(\mu\)};
    \vertex [right =1.5cm of m1] (m2);
    \vertex [right =1.8cm of m2] (m3);
    \vertex at ($(m2)!0.5!(m3)$) (w);
    \vertex at ($(m2)!0.5!(m3) + (0, 0.4cm)$) (wl) {\(W\)};
    \vertex at ($(m2)!0.5!(m3) + (0, 0.9cm)$) (n);
    \vertex [right =1.5cm of m3] (m4) {\(e\)};

    \vertex [below =of m1] (q1) {\(q\)};
    \vertex [below =of m2] (q2);
    \vertex [below =of m3] (q3);
    \vertex [below =of m4] (q4) {\(q\)};
    \vertex at ($(q2)!0.5!(q3)$) (qg);
    
    \diagram* {
    (m1) -- [fermion] (m2)
    -- [fermion, quarter left, edge label=\(\nu_\mu\)] (n)
    -- [fermion, quarter left, edge label=\(\nu_e\)] (m3),
    (m2) -- [boson] (w)
    -- [boson] (m3)
    -- [fermion] (m4),
    (q1) -- [fermion] (q2)
    -- (q3)
    -- [fermion] (q4),
    (w) -- [boson, edge label=\(\gamma\)] (qg),
    };

    \draw (n) -- (n) node {\(\times\)};
    
    \end{feynman}
    \end{tikzpicture}
    \caption{
        $\mu+N \rightarrow e + N$
    }
    \label{fig:mu-e_conv_SM}
\end{subfigure}
    \caption{
        Feynman diagrams of processes allowing charged lepton flavour
        violation in the SM extended with neutrino masses. Although these
        processes enable CLFV, their branching ratios are heavily suppressed to
        unobservable levels because of the lightness of
        neutrinos compared to the weak scale~\cite{BERNSTEIN201327}.}
\end{figure}

The fact that neutrinos can change flavour allows a process such as shown in
Fig.~\ref{fig:mu_e_nu_osc} to occur, which gives rise to a non-zero rate for
${\mu \rightarrow e + \gamma}$. However, the branching ratio calculated for this
process using the upper limit on neutrino masses is given by:
\begin{equation}\label{eq:br_meg}
\mathcal{BR}(\mu \rightarrow e + \gamma) = \frac{3\alpha}{32\pi} \left|\ \sum_{i=2, 3} U^*_{\mu i} U_{e i} 
\frac{\Delta m^2_{i1}}{m^2_W}  \ \right| ^2 \approx 10^{-54},
\end{equation}
where $U$ is the PMNS matrix, $\Delta m^2_{ij}$ is the mass-squared difference between the
$i$-th and $j$-th neutrino mass eigenstates, and $m_W$ is the $W$-boson
mass~\cite{BERNSTEIN201327}.
Hence, any evidence that the ${\mu \rightarrow e + \gamma}$ process occurs at a higher
rate than given by Eq.~\ref{eq:br_meg} would indicate that another channel,
involving charged lepton flavour violation, is responsible.

\section{Charged lepton flavour violation}

The processes involving muons which are most sensitive for probing charged
lepton flavour violation (CLFV) are ${\mu \rightarrow e + \gamma}$, ${\mu
\rightarrow e+e+e}$ and ${\mu + N \rightarrow e + N}$~\cite{BERNSTEIN201327}.
Observing CLFV with any one of them would corroborate the fact that the SM
Lagrangian is incomplete, as is now known from neutrino oscillations. Measuring
CLFV in two or more processes would then help us to determine which --- if any
--- of the many theorised models for new physics is realised in nature.


CLFV has been sought after ever since the muon's discovery: the first
investigation of whether nature allows $\mu \rightarrow e + \gamma$ was done in
1948~\cite{PhysRev.73.257}. A multitude of experiments followed, but none so far
have been able to find a signal. Fig.~\ref{fig:clfv_upper_limit} shows
experimentally-estimated upper limits on the branching ratios of $\mu
\rightarrow e\gamma$, $\mu\rightarrow eee$ and $\mu N \rightarrow e N$ over
time, since the first experiment and into the next decade. 

As higher and higher sensitivities are required, experiments must be able to
produce a muon source which is more and more intense while demonstrating a
precise control over every source of background. This is made possible by new
technologies --- in both hardware and software --- put in application across
entire experiment designs. The next generation of CLFV-seeking precision
experiments, which consists of MEG II, Mu3e, COMET and Mu2e, aims to be 10 to
\numprint{10000} times more sensitive than the last generation.

\begin{figure}
    \centering
    \includegraphics[width=\textwidth]{chapter1/clfv_upper_limit_v2.pdf}
    \caption{
        90\%-confidence upper limit on the branching ratio of three charged
        lepton flavour-violating processes over time. 
        The target material $N$ is indicated for $\mu$--$e$ conversion experiments.
        Past experiment results
        were tabulated in~\cite{BERNSTEIN201327}. Future data points are the
        expected sensitivities quoted in the MEG II~\cite{Baldini2018},
        Mu3e~\cite{ARNDT2021165679}, COMET
        Phase-I~\cite{the_comet_collaboration_comet_2020} and
        Mu2e~\cite{bartoszek2015mu2e} design reports.
    }
    \label{fig:clfv_upper_limit}
\end{figure}

\section{Muon to electron conversion}
% mu-e conversion
Muon to electron conversion is the neutrino-less decay of a muon bound to an
atomic nucleus:
$$
\mu^- + N(A, Z) \rightarrow e^- + N(A, Z),
$$
where $A$ is the mass number and $Z$ the atomic number.
Similarly to $\mu\rightarrow e+\gamma$, this process is allowed in the SM
extended with massive neutrinos via the diagram shown in
Fig.~\ref{fig:mu-e_conv_SM}, but suppressed to an experimentally
unreachable level. Any signs of it occurring at current experimental
sensitivities would suggest a BSM origin.

In order to search for this process, muons must be stopped in matter to form
muonic atoms. Initially bound in the outer atomic layers, the muon
electromagnetically cascades down to the $1s$ orbital within \SI{1}{\ns}
and gets in close range of the nucleus~\cite{Knecht2020}. 
When interacting with the nucleus, the reaction is called \emph{coherent} if the
nucleus remains in its ground state. For elements heavier than magnesium, the
ratio of coherent to incoherent $\mu$--$e$ conversion is expected to be around
9:1~\cite{CHIANG1993526}.

In a coherent $\mu$--$e$ conversion, the kinematics are those of a
straightforward two-body decay, hence the electron always has an energy
\begin{equation*}\label{eq:mu_e_conv_energy}
E_{\mu\text{--}e} = m_\mu - B_\mu - E_\mathrm{recoil},
\end{equation*}
where $B_\mu$ is the binding energy of the $1s$-state muonic atom and
$E_\mathrm{recoil}$ is the recoil energy of the nucleus. In aluminium, the
target material of the COMET and Mu2e experiments, this yields
$$
    E^\mathrm{Al}_{\mu\text{--}e} = \SI{104.97}{\MeV}.
$$
Since the signature of $\mu$--$e$ conversion is a single, mono-energetic
electron, this process should be relatively easily identified by means of a
momentum-measuring detector. However, this signal must also be discriminated
from backgrounds originating from other processes, contamination of the beam and
cosmic rays.

\subsection{Standard Model backgrounds}
In the SM, a bound muon is allowed to either decay in orbit: 
\begin{equation*}\label{eq:dio}
    \mu^- + N(A, Z) \rightarrow e^- + \nu_\mu + \overline{\nu}_e + N(A, Z),
\end{equation*}
or it can be captured by the nucleus via a $W$ exchange:
\begin{equation*}\label{eq:capture}
    \mu^- + N(A, Z) \rightarrow \nu_\mu + N(A, Z-1),
\end{equation*}
% Not great: RMC is the background, but non-radiative nuclear capture isn't.
% Make it clear.
% Might be better to split into two subsecs
% yeah but RMC is minor compared to DIO (x0.2)
% followed by a photon emission by the excited nucleus and $e^+e^-$-pair
% production.
Both decay in orbit and radiative muon capture can, depending on the target
material, produce electrons with an energy approaching $E_{\mu\text{--}e}$,
hence these events may be mistaken for a conversion electron. 

For an aluminium target, the most dangerous of the two processes is decay in
orbit (DIO) since the high-energy endpoint of the DIO spectrum is approximately
equal to $E^\mathrm{Al}_{\mu\text{--}e}$~\cite{czarnecki}. The rate of DIO in
the high energy tail is many orders of magnitude lower than at the peak, hence
these background events are expected to be extremely rare. However, the level of
sensitivity aimed at by next-generation searches makes DIO the most important
background source and thus one of the main sensitivity-limiting
factors~\cite{the_comet_collaboration_comet_2020}.

Sources of background other than these allowed SM processes are expected in
$\mu$--$e$ conversion-searching experiments, such as beam-related and cosmic
ray-induced events. Those specific to the COMET experiment, and the associated
design choices which were made to minimise their occurrence, are discussed in
Section~\ref{sec:backgrounds}.


% The choice of target material directly influences how frequently high-energy
% electrons will be produced by these processes, and hence contributes in
% setting the maximum experimental sensitivity.

% This should go into the COMET section as these are specific to COMET and/or
% specifically addressed by the COMET design.
% Background events can be classified into three categories: intrinsic,
% beam-related, and cosmic ray-induced.

% Beam-related backgrounds come from impurities in the high-intensity muon beam.
% The specifics of the COMET beam and how these backgrounds are alleviated is
% discussed in Section~\ref{sec:COMET_beam}.

% Cosmic rays typically produce muons with a wide range of energies in the
% atmosphere which 








% Since 1948~\cite{PhysRev.73.257}, searches for ${\mu \rightarrow e + \gamma}$
% have demonstrated that it does not occur at any observable rate: the current
% upper limit on its branching ratio measured by the MEG experiment is
% $10^{-13}$~\cite{mori2016final}.

% In the quark sector, generations can mix via weak interactions: the
% CKM matrix accurately describes how strongly each quark flavour couples
% to the others. %~\cite{PhysRevLett.10.531, 10.1143/PTP.49.652}.
% For leptons on the other hand, it is
% only through neutrino oscillations, discovered by the Super-Kamiokande
% experiment~\cite{PhysRevLett.81.1562}, that flavours have been observed to mix. 
% Flavour-changing neutrino oscillations allow a new channel for ${\mu \rightarrow
% e + \gamma}$, shown in Fig.~\ref{fig:mu_e_nu_osc}. 


% g-2 measurement
%\cite{PhysRevLett.126.141801} % g-2

% Is lhc-b related here? Lepton univ?
%\cite{Aaij2022}% lhc-b R_K




\section{Effective CLFV interactions}
CLFV searches are sensitive to various categories of new physics, including the
existence of an exotic Higgs, a $Z'$ boson, leptoquarks, heavy neutrinos, or
supersymmetric particles. Independently of the specific new physics responsible
for CLFV, we can consider a low-energy effective field theory to understand what
deviations future experiments might expect to see due to these new high-energy
degrees of freedom. 

\begin{figure}
    \centering
    \begin{subfigure}[b]{0.3\textwidth}
        \centering
        \feynmandiagram [small, inline=(b.base), vertical=b to d] {
            a [particle=\(\mu^{-}\)] -- [fermion] b [blob]
            -- [fermion] c [particle=\(e^-\)],
            b -- [boson] d [particle=\(\gamma\)],
        };
        \caption{Photonic}
    \end{subfigure}
    \hspace{1cm}
    \begin{subfigure}[b]{0.3\textwidth}
        \centering
        \feynmandiagram [small, inline=(b.base), vertical=a to d] {
            a [particle=\(\mu^{-}\)] -- [fermion] b [blob]
            -- [fermion] {c [particle=\(e^-\)],
            e [particle=\(q\)]},
            d [particle=\(q\)] -- [fermion] b,
        };
        \caption{Four-fermionic}
    \end{subfigure}
    \caption{New tree-level vertices which arise in the effective Lagrangian
    of Eq.~\ref{eq:Leff} and allow CLFV.}
    \label{fig:tree_lvl_clfv}
\end{figure}

\hl{Yeah but what's deGouvea's starting point?}\\
After integrating out heavy fields (see
e.g.~\cite[Chapter~IV]{donoghue_golowich_holstein_2014}), one obtains the
following effective Lagrangian, which allows CLFV to be mediated by the
tree-level vertices shown in Fig.~\ref{fig:tree_lvl_clfv}:
\begin{align}\label{eq:Leff}
    \mathcal{L^\text{eff}_\mathrm{CLFV}}\,
    =\,
    &\frac{1}{\kappa+1}\, \frac{m_\mu}{\Lambda^2}\,
    \overline{\mu}_R \sigma_{\mu\nu} e_L \, F^{\mu\nu} + \text{h.c.}
    \nonumber\\[1em]
    +\,
    &\frac{\kappa}{\kappa+1}\, \frac{1}{\Lambda^2}\,
    \overline{\mu}_L \gamma_\mu e_L \,
    (\,
        \overline{u}_L \gamma^\mu u_L + \overline{d}_L \gamma^\mu d_L
    \,) + \text{h.c.}
\end{align}
where $\Lambda$ is the energy scale of the new physics, and $\kappa$ determines
whether the preferred channel is photonic ($\kappa \rightarrow 0$) or
four-fermionic ($\kappa \rightarrow \infty$)~\cite{DEGOUVEA201375}. 
From the sensitivity of a future experiment, we can estimate the maximum scale
of new physics $\Lambda$ which the experiment will be able to probe. 
% If a new
% particle exists with a mass lower than that, the experiment should see a
% significant deviation from expectation, i.e. it should observe CLFV.
Specifically, COMET and Mu2e, which have a single-event sensitivity of around
$10^{-17}$, will probe energy scales up to $10^4$
\si{\tera\eV}~\cite{DEGOUVEA201375}.

If CLFV is observed in one of the three muon channels, the value of $\kappa$ can
then be determined from a measurement of the branching ratio of a second
channel. This information can then be used to constrain models and pinpoint the
true nature of the new physics.

The most accurate measurement of the muon magnetic moment to date from the Muon
$g-2$ experiment shows a deviation of $4.2\sigma$ from the Standard Model
prediction~\cite{PhysRevLett.126.141801}. The energy scale of new physics
associated with the measured anomaly is expected to lie below
\SI{2.1}{\tera\eV}~\cite{KESHAVARZI2022115675}. Following~\cite{DEGOUVEA201375},
we can define a parameter $\theta_{e\mu} =
\frac{\Lambda^2_{g-2}}{\Lambda^2_\text{CLFV}}$ which measures how flavour-conserving
is the new physics responsible for the anomalous $g-2$. Assuming that this
anomaly is caused by high-energy BSM contributions, the current CLFV upper
limits set the maximum value for $\theta_{e\mu}$ at $10^{-5}$, i.e.\ the
amount of flavour violation in the new physics sector is very small.


\hl{This feels stupid... Why even include it? Doesn't sound very relevant tbh.}