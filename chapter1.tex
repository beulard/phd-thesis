\chapter{Lepton Flavour in the Standard Model and Beyond}\label{chapter1}

% \begin{markdown}
% ---

% + Yoshi said: this is really an intro to CLFV searches for experts out of the CLFV
% field.

% + What's a muon?
% + What does the SM predict muons can/can't do
% + Historical context, motivation
%  + Why/how was the first CLFV experiment conducted?


% ---
% \end{markdown}

% v1
The Standard Model (SM) of particle physics is possibly the most successful
mathematical model of physical phenomena so far. It provides an accurate
description of all observable interactions between known elementary particles.
It yields predictions for what nature does and does not allow, and enables
physicists to examine and test the fundamental laws of the universe. It provides
a rigid framework to make theoretical predictions, from which any measured
deviations can then be interpreted as new physics.

The muon is the most sensitive probe to search for charged lepton flavour
violation. If that is observed, it will add another piece of definite evidence,
along with non-zero neutrino masses, that the SM cannot faithfully account for
all fundamental interactions.


% However before we are able to dive into the main topics, it is necessary to
% discuss the purpose of the experiment, how it emerges from the current state of
% our knowledge of elementary particle physics, and the steps taken to get there.


% v0
% Explain the SM. Something like
% The Standard Model is the theory at the heart of modern particle physics. Built
% upon special relativity and quantum mechanics, it describes the interactions
% between elementary particles and allows physicists to predict the outcome of
% interactions and decays to a previously unattainable precision. Little evidence
% so far has been able to contradict the formulation of the Standard Model,
% despite many fundamental questions remaining unanswered, such as the nature of
% dark matter, the reason for the matter-antimatter asymmetry in the universe, or
% the existence of exactly three generations of leptons and quarks.


\section{Discovery of the muon}
The first traces of muons were observed around 1937 by three experiments
investigating the nature of cosmic ray-induced particle
showers~\cite{PhysRev.51.884, PhysRev.52.1198, PhysRev.52.1003}. 
One of them was able to estimate the mass of the discovered
particle at 130 times that of the electron, around the same as the strong
force-carrying meson predicted by Yukawa in 1935~\cite{10.1143/PTPS.1.1}. 
Hence the muon and Yukawa's particle were
originally believed to be one and the same particle, and it was only when
the meson (now called $\pi$, for primary) was observed decaying into a muon,
that the two particles were completely disambiguated~\cite{LATTES1947}.

It was the fact that the muon appeared as nothing but a heavy electron which
prompted Rabi to ask ``who ordered that?'' in response to its discovery. In
fact, the Standard Model still cannot give a satisfactory answer to this
question as it does not motivate the existence of three generations of
elementary particles. As far as the SM can explain, there is no fundamental reason
for the existence of distinct flavours.

\section{The muon in the Standard Model}
% 
The SM identifies the muon as the second-generation charged lepton, meaning it
is a fermion with identical quantum numbers (aside from flavour) as the electron
and tau lepton. The only way for a muon to decay in a vacuum is through the weak
force. The diagram for muon decay, $\mu^- \rightarrow e^- +  \nu_\mu
+ \overline{\nu}_e$, is shown in Fig.~\ref{fig:weak_decay}.


\begin{figure}
    \centering
    \feynmandiagram [layered layout, horizontal=a to b] {
        a [particle=\(\mu^{-}\)] -- [fermion] b -- [fermion] f1 [particle=\(\nu_{\mu}\)],
        b -- [boson, edge label'=\(W^{-}\)] c,
        c -- [anti fermion] f2 [particle=\(\overline \nu_{e}\)],
        c -- [fermion] f3 [particle=\(e^{-}\)],
    };
    \caption{Feynman diagram for the weak decay of the muon.}
    \label{fig:weak_decay}
\end{figure}

In the SM Lagrangian with massless neutrinos, none of the terms which involve
leptons allow for flavour violation. The Lagrangian is invariant under the
$U(1)_e \times U(1)_\mu \times U(1)_\tau$ group, and consequently, each lepton
family ($e$, $\mu$, $\tau$) has its own conserved number which prevents a
charged lepton from changing flavour without neutrinos being involved. 

These conservation laws do not correspond to a fundamental symmetry of nature;
they are merely a feature of the SM Lagrangian and have, so far, been observed
to hold experimentally. For instance, the process ${\mu \rightarrow e +
\gamma}$, in principle allowed by kinematics, has never been observed and the
current upper limit on its branching ratio was set by the MEG experiment at
$10^{-13}$~\cite{mori2016final}.

The observation of neutrino oscillations~\cite{PhysRevLett.81.1562} means that
the three accidental flavour symmetries are not exact. This strongly
motivates a search for flavour violation among the charged leptons as well, as
any evidence for it would, together with neutrino oscillations, yield hints
about the theory lying beyond the Standard Model.

The fact that neutrinos can change flavour allows a process such as shown in
Fig.~\ref{fig:mu_e_nu_osc} to occur, which gives rise to a non-zero rate for
${\mu \rightarrow e + \gamma}$. However, the branching ratio calculated for this
process using the upper limit on neutrino masses is given by:
\begin{equation}\label{eq:br_meg}
\mathcal{BR}(\mu \rightarrow e + \gamma) = \frac{3\alpha}{32\pi} \left|\ \sum_{i=2, 3} U^*_{\mu i} U_{e i} 
\frac{\Delta m^2_{i1}}{M^2_W}  \ \right| ^2 \approx 10^{-54},
\end{equation}
where $U$ is the PMNS matrix, $\Delta m^2_{ij}$ is the mass-squared difference between the
$i$-th and $j$-th neutrino mass eigenstates, and $M_W$ is the $W$-boson
mass~\cite{BERNSTEIN201327}.
Hence any evidence that the ${\mu \rightarrow e + \gamma}$ process occurs at a higher
rate than given by Eq.~\ref{eq:br_meg} would indicate that another channel,
involving charged lepton flavour violation (CLFV), is responsible.

\section{Charged lepton flavour violation}
\hl{HISTORY}

Since 1948~\cite{PhysRev.73.257}, searches for ${\mu \rightarrow e + \gamma}$
have demonstrated that it does not occur at any observable rate: the current
upper limit on its branching ratio measured by the MEG experiment is
$10^{-13}$~\cite{mori2016final}.

In the quark sector, generations can mix via weak interactions: the
CKM matrix accurately describes how strongly each quark flavour couples
to the others. %~\cite{PhysRevLett.10.531, 10.1143/PTP.49.652}.
For leptons on the other hand, it is
only through neutrino oscillations, discovered by the Super-Kamiokande
experiment~\cite{PhysRevLett.81.1562}, that flavours have been observed to mix. 
Flavour-changing neutrino oscillations allow a new channel for ${\mu \rightarrow
e + \gamma}$, shown in Fig.~\ref{fig:mu_e_nu_osc}. 

\begin{figure}
    \centering
    \begin{tikzpicture}
    \begin{feynman}
    \vertex (a) {\(\mu\)};
    \vertex [right=1.5cm of a] (b);
    \vertex [right=1.8cm of b] (c);
    \vertex [right=1.5cm of c] (d) {\(e\)};
    
    \vertex at ($(b)!0.5!(c) + (0, 0.9cm)$) (n);
    \vertex at ($(b)!0.5!(c) + (0, -0.9cm)$) (w);
    \vertex [above=0.1cm of w] (wn) {\(W\)};
    \vertex [below=1.5cm of d] (g) {\(\gamma\)};
    
    \diagram* {
    (a) -- [fermion] (b)
    -- [fermion, quarter left, edge label=\(\nu_\mu\)] (n)
    -- [fermion, quarter left, edge label=\(\nu_e\)] (c),
    (b) -- [boson, quarter right] (w)
    -- [boson, quarter right] (c),
    (c) -- [fermion] (d),
    (w) -- [boson] (g),
    };
    
    \draw (n) -- (n) node {\(\times\)};
    
    \end{feynman}
    \end{tikzpicture}
    \caption{
        Feynman diagram for $\mu \rightarrow e + \gamma$ via
        flavour oscillation of a virtual neutrino.
    }
    \label{fig:mu_e_nu_osc}
\end{figure}



In the presence of matter, there are additional processes which could be
investigated to find charged lepton flavour violation (CLFV), as shown in Fig.


% Allowed and forbidden decays

% g-2 measurement
\cite{PhysRevLett.126.141801} % g-2

% Is lhc-b related here? Lepton univ?
\cite{Aaij2022}% lhc-b R_K




\section{Charged lepton flavour violation}
Quark generations are known to mix (CKM).
Neutral leptons (neutrinos) are known to mix (PMNS, osc).
Charged leptons have never been seen to mix. Why?

\begin{figure}
    \centering
    \includegraphics[width=\textwidth]{chapter1/clfv_upper_limit_v2.pdf}
    \caption{
        90\%-confidence upper limit on the branching ratio of three charged
        lepton flavour-violating processes over time. Past experiment results
        were tabulated in~\cite{BERNSTEIN201327}. Future data points are the
        expected sensitivities quoted in the MEG II~\cite{Baldini2018},
        Mu3e~\cite{ARNDT2021165679}, COMET
        Phase-I~\cite{the_comet_collaboration_comet_2020} and
        Mu2e~\cite{bartoszek2015mu2e} design reports.
    }
    \label{fig:clfv_upper_limit}
\end{figure}
